\documentclass[12pt,a4paper]{article}

\usepackage[MeX]{polski}
\usepackage[utf8]{inputenc}
\usepackage[T1]{fontenc}
\usepackage{fullpage}
\usepackage{textcomp}
\usepackage{amsmath, amssymb}
\usepackage{minted}
\usepackage{titling}
\setlength\parindent{0pt}

\newcommand{\subtitle}[1]{%
  \posttitle{%
    \par\end{center}
    \begin{center}\large#1\end{center}
    \vskip0.5em}%
}

\newcommand\nt[1]{\mkern1mu\overline{\mkern-1mu#1\mkern-1mu}\mkern1mu}

\author{Krzysztof Ruczkowski}
\title{Raport}
\subtitle{System ewidencji czasu pracy\\wykorzystujący technologie Internetu Rzeczy}


\begin{document}

\maketitle
\newpage
\tableofcontents
\newpage

\section{Dane raportu}

\begin{tabular}{ll}
Przedmiot: & Podstawy Internetu Rzeczy\\
Imię i nazwisko autora: & Krzysztof Ruczkowski\\
Nr indeksu: & 246739\\
Semestr studiów: & 4\\
Data ukończenia pracy: & Maj 2020 r.\\
Prowadzący labolatorium: & mgr inż. Piotr Jóźwiak\\
\end{tabular}

\section{Wymagania projektowe}

\subsection{Wymagania funkcjonalne}

\begin{enumerate}
  \item Rejestracja i usuwanie danych terminali odczytujących karty
  \item Rejestracja i usuwanie danych pracowników
  \item Rejestracja i usuwanie danych kart RFID
  \item Przypisywanie i usuwanie przypisań pracowników do kart RFID
  \item Rejestrowanie odczytów kart RFID
  \item Generowanie pliku csv zawierającego czasy przychodzenia i wychodzenia z pracy pracowników
\end{enumerate}

\subsection{Wymagania niefunkcjonalne}
\begin{enumerate}
  \item Użycie Pythona 3 do implementacji
  \item Użycie MQTT do komunikacji
  \item Użycie tkinter do wyświetlania GUI
  \item Użycie bazy danych SQLite do trwałego przechowania danych
  \item Bezpieczny przesył danych (autoryzacja i uwierzytelnianie)
\end{enumerate}

\section{Opis architektury systemu}

System (\texttt{server.py}) to aplikacja nasłuchująca uwierzytelnionych wiadomości przesyłanych przez MQTT zawierającyh informacje dotyczące odczytu karty.
Aplikacja jest wyposażona w GUI, za pomocą którego możemy spełnić resztę wymagań funkcjonalnych.\\
Możemy zasymulować odczyt karty za pomocą \texttt{client.py}.

\section{Opis implementacji i zastosowanych rozwiązań}
\subsection{Baza danych}
Została wykorzystana baza danych SQLite, przykład z pliku \texttt{src/appdata.py}:
\begin{minted}{python}
def fetchall(sql):
    return c.execute(sql).fetchall()

def get_employees():
    return fetchall("SELECT * FROM Employee")

def get_cards():
    return fetchall("SELECT * FROM Card")

def get_card_readings():
   return fetchall(
"""
SELECT
Employee.id, Employee.name, Terminal.name, readTime
FROM CardReading
INNER JOIN Card ON CardReading.rfid = Card.rfid
INNER JOIN Terminal ON CardReading.terminalId = Terminal.id
LEFT JOIN Employee ON Card.employeeId = Employee.id
""")

def get_terminals():
    return fetchall("SELECT * FROM Terminal")

def insert_employee(name):
    c.execute("INSERT INTO Employee (name) VALUES (?)", (name,))
    conn.commit()
\end{minted}

\subsection{Generacja pliku csv z czasami}
Za generację pliku odpowiada funkcja w \texttt{src/app/controller.py}:

\begin{minted}{python}
def generate_time_report():
    logger.log("Generating time report...")

    cr = data.get_card_readings()

    employee_ids_on_readings = set(x[0] for x in cr)
    cr_by_employee_id = {y:[(x[2], x[3]) for x in cr if x[0] == y] for y in employee_ids_on_readings}
    cr_by_employee_id_paired = {k:list(zip(v[::2], v[1::2])) for (k, v) in cr_by_employee_id.items()}
    i = cr_by_employee_id_paired.items()

    csv = "\n".join([",".join([str(k or "None"), t[0][0], t[0][1], t[1][0], t[1][1]]) for (k, v) in i for t in v])

    with open("time_report.csv", "w") as file:
        file.write(csv)

    logger.log("Time report generated.")
\end{minted}
Może wyglądać nieco skomplikowanie, dlatego skupmy się na nazwach zmiennych.\\
Najpierw wyciągamy dane z bazy odnośnie odczytów kart do zmiennej \texttt{cr}.
Operujemy na tej zmiennej, najpierw wyciągając zbiór ID pracowników na odczytach.
Następnie tworzymy słownik, którego kluczem jest ID pracownika, a wartością jego odczyty z karty.
Potem wystarczy połączyć odczyty w pary, gdzie pierwszy odczyt jest czasem przyjścia do pracy a drugi czasem wyjścia, sformatować te pary do formatu csv i zapisać plik.

\section{Opis działania i prezentacja interfejsu}
\subsection{Przygotowanie środowiska}
\subsubsection{Konfiguracja Mosquitto}
Należy skonfigurować MQTT dodając uwierzytelnianie i autoryzację, tworząc użytkowników 'client' i 'server' oraz używając certyfikatów z folderu \texttt{certs}.\\

Zawartość aclfile.conf:

\begin{minted}{bash}
user server
topic read app/card_reading

user client
topic app/card_reading
\end{minted}
\subsubsection{Pobranie zależności Python'a}
\begin{minted}{bash}
cd src
./setupenv.sh
source venv/bin/activate
\end{minted}

\subsection{Uruchomienie aplikacji}
Sprawdzić działanie aplikacji można uruchamiając serwer i klientów symulujących terminale:
\begin{minted}{bash}
./server.py &; ./client.py &; ./client.py &;
\end{minted}

\subsection{Screeny}
Screeny przedstawiające działanie aplikacji znajdują się w folderze \texttt{screens}.

\section{Podsumowanie}
Projekt jest zgodny ze wszystkimi wymaganiami funkcjonalnymi.\\
Trudnością w implementacji było użycie tkinter'a z bazą danych, której trzeba otwarcie umożliwić działanie na wielu wątkach. Samo zdebugowanie tego problemu wymagało czasu ze względu na domyślne ignorowanie wyjątków przez tkinter (dopiero użycie pdb (python debugger) pomogło).\\
Rozbudowa projektu nie powinna stanowić problemu ze względu na modularność i rozdzielenie odpowiedzialności między moduły.
\section{Aneks}
Kod projektu znajduje się na eportalu.

\end{document}
