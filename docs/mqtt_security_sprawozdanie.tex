\documentclass[10pt,a4paper]{article}

\author{Krzysztof Ruczkowski}
\title{Sprawozdanie\\bezpieczeństwo protokołu MQTT --- wprowadzenie}

\usepackage[MeX]{polski}
\usepackage[utf8]{inputenc}
\usepackage[T1]{fontenc}
\usepackage{fullpage}
\usepackage{textcomp}
\usepackage{amsmath, amssymb}
\usepackage{minted}

\newcommand\nt[1]{\mkern1mu\overline{\mkern-1mu#1\mkern-1mu}\mkern1mu}

\begin{document}
\maketitle
\newpage
\tableofcontents
\newpage

\section{Wprowadzenie}

Sprawozdanie dotyczy labolatorium wprowadzającego z bezpieczeństwa protokołu MQTT.
\\
Przetestowane na systemie Arch Linux.

\section{Przebieg prac}

\subsection{Instalacja Mosquitto i Openssl}
Wystarczy zainstalować paczki \texttt{mosquitto} i \texttt{openssl}, a następnie uruchomić serwis \texttt{mosquitto}:

\begin{minted}{bash}
  pacman -S mosquitto openssl
  systemctl enable --now mosquitto
\end{minted}

\subsection{Generacja certyfikatów}
Używając \texttt{openssl} można wygenerować certyfikaty wymagane do skonfigurowania połączenia TLS:

\begin{minted}{bash}
openssl genrsa -des3 -out ca.key 2048
openssl req -new -x509 -days 1826 -key ca.key -out ca.crt
openssl genrsa -out server.key 2048
openssl req -new -out server.csr -key server.key
# tutaj należy wpisać nazwę domeny serwera w polu "Common Name"
openssl x509 -req -in server.csr -CA ca.crt -CAkey ca.key -CAcreateserial -out server.crt -days 360
\end{minted}

\subsection{Konfiguracja Mosquitto}
Po wygenerowaniu certyfikatów należy edytować plik \texttt{mosquitto.conf} zgodnie z instrukcją.
Należy dodać tam port i ścieżki do certyfikatów.

\begin{minted}{bash}
vim /etc/mosquitto/mosquitto.conf
\end{minted}

Po konfiguracji nie trzeba uruchamiać ponownie komputera, wystarczy zrestartować serwis:

\begin{minted}{bash}
systemctl restart mosquitto
\end{minted}

Tutaj może pojawić się błąd z uruchomieniem usługi. Należy upewnić się, że serwis ma uprawnienia do odczytu certyfikatów, a jeśli nie ma to dodać je poleceniem \texttt{chmod}.

\subsection{Test konfiguracji}
Po konfiguracji pozostaje przetestować funkcjonalność przykładu, odpowiednio modyfikując pliki \texttt{sender.py} i \texttt{receiver.py}:

\begin{minted}{python}
broker = "MightyTos4"
port = 8883

client.tls_set("ca.crt")
client.connect(broker, port)
\end{minted}

\section{Podsumowanie}
Wykonanie ćwiczenia na Linuksie było proste i przyjemne.\\
Na Windowsie to także jest proste i wygląda analogicznie - trzeba tylko ręcznie instalować Mosquitto i Openssl, pamiętając o dodaniu binarek do zmiennej środowiskowej \texttt{PATH} dla wygody. Do uruchomienia i restartu serwisu Mosquitto także ponowne uruchomienie systemu nie jest wymagane, jak sugeruje instrukcja - wystarczy znaleźć serwis "Mosquitto Broker" w \texttt{services.msc}.

\end{document}
