\documentclass[10pt,a4paper]{article}
\author{Krzysztof Ruczkowski}
\title{Sprawozdanie\\bezpieczeństwo protokołu MQTT\\uwierzetylnianie i autoryzacja}

\usepackage[MeX]{polski}
\usepackage[utf8]{inputenc}
\usepackage[T1]{fontenc}
\usepackage{fullpage}
\usepackage{textcomp}
\usepackage{amsmath, amssymb}
\usepackage{minted}

\setlength\parindent{0pt}

\newcommand\nt[1]{\mkern1mu\overline{\mkern-1mu#1\mkern-1mu}\mkern1mu}
\begin{document}
\maketitle
\newpage
\tableofcontents
\newpage

\section{Wprowadzenie}

Sprawozdanie dotyczy labolatorium z bezpieczeństwa protokołu MQTT odnoszącego się do uwierzytelniania i autoryzacji.
\\
Przetestowane na systemie Arch Linux.

\section{Przebieg prac}

\subsection{Stworzenie użytkowników}
Poniższe komendy wykonuję jako superuser, aby móc zapisywać do katalogu \texttt{/etc/mosquitto/}:

\begin{minted}{bash}
  cd /etc/mosquitto/
  mosquitto_passwd -c passwd.conf server
  mosquitto_passwd -b passwd.conf client password
\end{minted}

Plik \texttt{passwd.conf} zawiera loginy i zaszyfrowane hasła w formacie przypominającym plik \texttt{/etc/shadow}.

\subsection{Konfiguracja uwierzytelniania}
Po stworzeniu użytkowników należy edytować plik \texttt{mosquitto.conf} zgodnie z instrukcją.
Należy dodać tam linijki uniemożliwiające połączenie bez loginu (\texttt{allow\_anonymous false}) oraz wskazać plik z loginami i hasłami (\texttt{password\_file /etc/mosquitto/passwd.conf}).

\begin{minted}{bash}
  vim /etc/mosquitto/mosquitto.conf
\end{minted}

Po konfiguracji nie trzeba uruchamiać ponownie komputera, wystarczy zrestartować serwis:

\begin{minted}{bash}
  systemctl restart mosquitto
\end{minted}

\subsection{Test konfiguracji uwierzytelniania}
Po konfiguracji pozostaje przetestować funkcjonalność przykładu, odpowiednio modyfikując pliki \texttt{sender.py} i \texttt{receiver.py}:

\begin{minted}{python}
# dane dostępowe dla klienta
# analogicznie ustawiamy dane dostępowe serwera w receiver.py
client.username_pw_set(username='client', password='password')
\end{minted}

Jeżeli nie dodamy uwierzytelnienia, nie będzie można wysyłać i odbierać wiadomości.

\subsection{Stworzenie pliku kontroli dostępu}
Poniższe komendy wykonuję jako superuser, aby móc zapisywać do katalogu \texttt{/etc/mosquitto/}:

\begin{minted}{bash}
  cat > /etc/mosquitto/acfile.conf
  # This only affects clients with username "server".
  user server
  topic server/name
  topic worker/name

  # This only affects clients with username "client".
  user client
  topic read server/name
  topic worker/name
\end{minted}

Działanie tego pliku jest opisane w sekcji \texttt{Default authentication and topic access control} w \texttt{mosquitto.conf}.

\subsection{Konfiguracja autoryzacji}
Po stworzeniu użytkowników należy edytować plik \texttt{mosquitto.conf} zgodnie z instrukcją.
Należy dodać tam wskazać plik kontroli dostępu (\texttt{acl\_file /etc/mosquitto/acfile.conf}).

\begin{minted}{bash}
  vim /etc/mosquitto/mosquitto.conf
\end{minted}

Po konfiguracji nie trzeba uruchamiać ponownie komputera, wystarczy zrestartować serwis:

\begin{minted}{bash}
  systemctl restart mosquitto
\end{minted}

\subsection{Test konfiguracji autoryzacji}
Po konfiguracji pozostaje przetestować funkcjonalność przykładu, odpowiednio modyfikując pliki \texttt{sender.py} i \texttt{receiver.py}:

\begin{minted}{python}
# sender.py
from tkinter import messagebox

# (...)
def process_message(client, userdata, message):
    message_decoded = (str(message.payload.decode("utf-8")))
    messagebox.showinfo("Message from the Server", message_decoded)

def connect_to_broker():
    # (...)
    client.on_message = process_message
    client.loop_start()
    client.subscribe("server/name")
    # (...)

def disconnect_from_broker():
    # (...)
    client.loop_stop()
    # (...)
\end{minted}
\begin{minted}{python}
# receiver.py
# (...)
def create_main_window():
    # (...)
    hello_button = tkinter.Button(window, text="Hello from the server",
            command=lambda: client.publish("server/name", "Hello from the server"))
    # (...)
    hello_button.pack(side="right")
    # (...)
\end{minted}

\section{Podsumowanie}
Uwierzytelnianie i autoryzacja w MQTT pozwala nam na bezpieczny przesył danych, aby uniemożliwić niepowołanym osobom na ich przechwycenie.
\\\\
Można nasłuchiwać wszystkich wiadomości używając komendy \texttt{mosquitto\_sub}, co przydaje się w debugowaniu:
\begin{minted}{bash}
mosquitto_sub -v -h MightyTos4 -t '#' -u server -P server_pass -p 8883 --cafile ca.crt
\end{minted}
\end{document}
